\documentclass[11pt,a4wide]{article}
\usepackage{verbatim}
\usepackage{listings}
\usepackage{graphicx}
\usepackage{a4wide}
\usepackage{color}
\usepackage{amsmath}
\usepackage{amssymb}
\usepackage[dvips]{epsfig}
\usepackage[T1]{fontenc}
\usepackage{cite} % [2,3,4] --> [2--4]
\usepackage{shadow}
\usepackage{hyperref}

\setcounter{tocdepth}{2}

\usepackage{subcaption}

\lstset{language=c++}
\lstset{alsolanguage=[90]Fortran}
\lstset{basicstyle=\small}
\lstset{backgroundcolor=\color{white}}
\lstset{frame=single}
\lstset{stringstyle=\ttfamily}
\lstset{keywordstyle=\color{red}\bfseries}
\lstset{commentstyle=\itshape\color{blue}}
\lstset{showspaces=false}
\lstset{showstringspaces=false}
\lstset{showtabs=false}
\lstset{breaklines}



%lager heftig forside:
\newcommand*{\titleAT}{\begingroup % Create the command for including the title page in the document
\newlength{\drop} % Command for generating a specific amount of whitespace
\drop=0.1\textheight % Define the command as 10% of the total text height

\rule{\textwidth}{1pt}\par % Thick horizontal line
\vspace{2pt}\vspace{-\baselineskip} % Whitespace between lines
\rule{\textwidth}{0.4pt}\par % Thin horizontal line

\vspace{0.5\drop} % Whitespace between the top lines and title
\centering % Center all text
\textcolor{black}{ % Red font color
{\Huge Studying second order phase transitions by using the Ising model in two dimentions and the metropolis algorithm}\\[0.75\baselineskip] % Title line 1
%{\Large Tema:}\\[0.75\baselineskip] % Title line 2
%{\Huge Lydmåling og hørselstesting} % Title line 3
} 

\vspace{0.25\drop} % Whitespace between the title and short horizontal line
\rule{0.3\textwidth}{0.4pt}\par % Short horizontal line under the title
\vspace{0.25\drop} % Whitespace between the thin horizontal line and the author name

{\Large \textsc{Project 4, FYS-3150\\[0.75\baselineskip] \normalsize{Ina K. B. Kullmann}
}}\par % Author name

%\vfill % Whitespace between the author name and publisher text

\vspace{0.25\drop} % Whitespace between the title and short horizontal line
\rule{0.3\textwidth}{0.4pt}\par % Short horizontal line under the title
\vspace{0.25\drop} % Whitespace between the thin horizontal line and the author name

\begin{abstract}
The aim of this project is to numerically solve ..... by using the .... algorithm. 


the Ising model in two dimensions, without an external magnetic 
field

Title: Studying phase transitions (critical T) using the ising model (metropolis alg)?

compare teory, lars onsager



%Before solving the problem for two electrons we will look at a simpler system, one electron in a three-dimensional harmonic oscillator potential. Then we will move on to the two electrons in a three-dimensional harmonic oscillator, but first without the Coulomb interaction. For each case we will reformulate the Schr\"odingers equation for this system to a dimentionless form and then to a descrete eigenvalue equation. This eigenvalue problem will we solve numerically with the Jacobi rotation algorithm and compare with the Armadillo library functions for the one electron system. 

%The Jacobi rotation algorithm will be implemented numerically on a general form so that it can be applied to any eigenvalue problem with a symetrix matix. We will test the implementation of the Jacobi method on an arbitrary symetric matrix before moving on to solve the physical problems. 

%When the procedure used on the two simplest cases are tested and understood we will apply the metods on the two electron system with a Coulomb interaction and plot the probability distrubution for different strengths of the interaction. 
\end{abstract}
\vspace*{0.25\drop} % Whitespace under the publisher text

\begin{center}
{ \scriptsize \noindent All source codes can be found at: \texttt{https://github.com/inakbk/Project\_2}. }
\end{center}

\rule{\textwidth}{0.4pt}\par % Thin horizontal line
\vspace{2pt}\vspace{-\baselineskip} % Whitespace between lines
\rule{\textwidth}{1pt}\par % Thick horizontal line

\endgroup}
%kode slutt for heftig forside


\begin{document}
%\maketitle
\titleAT % This command includes the title page


\newpage
\tableofcontents
\newpage

\section{Motivation and purpose}


\section{Theory}
something about phase transitions and therefore chi and heat capacity is relevant


\subsection{The Ising model in two dimentions}

The Ising model is a simple model for ferromagnetism in statistical physics. The model consists of magnetic spins that are allowed to interact with its neigbours. The magnetic dipole moments are allowed to have two values, 1 and -1.

In this project we will study the ising model i two dimentions wich allows the identification of phase transitions (define). We will also set the external magnetic field to zero troughout this paper.
% in two dimensions second order phase transition. 

%\subsection{Phase transitions} %?


The Ising model gives that the energy (for the whole system?) can be expressed as
\begin{equation}
  E=-J\sum_{<kl>}^{N}s_ks_l
\end{equation}
where the value of the spins are  $s_k=\pm 1$ and $N$ is the total number of spins. The variable $J$ is a coupling constant expressing the strength of the interaction between neighboring spins.

The symbol $<kl>$ indicates that we sum over nearest neighbors only. We will assume that we have a ferromagnetic ordering, viz $J> 0$ and use periodic boundary conditions. 

(invlude matrix of spins and define variable $L$, $N$ and $M$)

%analytical values to test:
First we assume that there is only two spins in each dimention (x and y), we set $L=2$. Then the closed form expression for the partition function is given by:
\[
Z = \sum_{i=1}^M e^{-\beta E_i}
\]
where $M$ is the number of microstates or combinations of spins.


\subsection{L=2 analytical values}

$N = L\cdot L = 4$

There is $M= L^N = 2^N = 16$ microstates, or possible combinations and energies of the spin system. So to calculate the partition function we have to find the energies:
\[
Z = e^{-\beta E_1} + ... e^{-\beta E_{16}}
\]

The spin system can be visualized as 16 matrices on the form:
$\left[ \begin{array}{cc} s(0,0) & s(0,1) \\
                             	    s(1,0)  & s(1,1) \\
\end{array} \right]$
with corresponding energies. 

Periodic boundaryconditions give that every spin has a neighbour. The neighbours of the spin $s(1,1)$ is  $s(0,1)$ twice (above and below) and $s(1,0)$ twice (above and below). To find the energy we have to sum over the nearest neighbours for all the spins in the system. The product $s_ls_k$ of two neighbours should only be calculated once. This is solved if the index of $s_l$ is fixed while $s_k$ has one higher index in each direction at a time:

\begin{align}
E &= -J\bigg( s(0,0)\cdot \big[s(1,0) + s(0,1) \big] + s(0,1)\cdot\big[s(1,1) + s(0,0) \big] \nonumber   \\ 
 &+ s(1,0)\cdot\big[s(0,0) + s(1,1) \big] + s(1,1)\cdot\big[s(0,1) + s(1,0) \big] \bigg)
 \label{eq: energy spins}
\end{align}

The 16 states with corresponding energies calculated with equation \ref{eq: energy spins} is then: \\
\begin{tabular}{cccc}
$E_1 = 0$ & $E_2 = 0$ & $E_3 = 0$& $E_4 = 0$ \\ 
$\left[ \begin{array}{cc} 1 & 1 \\
                             	    -1  & -1 \\
\end{array} \right]$ & 
$
\left[ \begin{array}{cc} 1 & -1 \\
                             	    1  & -1 \\
\end{array} \right]$ &
$
\left[ \begin{array}{cc} -1 & 1 \\
                             	    -1  & 1 \\
\end{array} \right]$ & 
$
\left[ \begin{array}{cc} -1 & -1 \\
                             	    1  & 1 \\
\end{array} \right] $ \\
$E_5 = 0$ & $E_6 = 0$ & $E_7 = 0$& $E_8 = 0$ \\ 
$
\left[ \begin{array}{cc} 1 & 1 \\
                             	    1  & -1 \\
\end{array} \right]$ &
$
\left[ \begin{array}{cc} 1 & 1 \\
                             	    -1  & 1 \\
\end{array} \right]$ &
$
\left[ \begin{array}{cc} 1 & -1 \\
                             	    1  & 1 \\
\end{array} \right]$ &
$
\left[ \begin{array}{cc} -1 & 1 \\
                             	    1  & 1 \\
\end{array} \right]$ \\
$E_9 = 0$ & $E_{10} = 0$ & $E_{11} = 0$& $E_{12} = 0$ \\ 
$
\left[ \begin{array}{cc} -1 & -1 \\
                             	    -1  & 1 \\
\end{array} \right]$ &
$
\left[ \begin{array}{cc} -1 & -1 \\
                             	    1  & -1 \\
\end{array} \right]$ &
$
\left[ \begin{array}{cc} 1 & -1 \\
                             	    -1  & -1 \\
\end{array} \right]$ &
$
\left[ \begin{array}{cc} -1 & 1 \\
                             	    -1  & -1 \\
\end{array} \right]$ \\
$E_{13} = 8J$ & $E_{14} = 8J$ & $E_{15} = -8J$& $E_{16} = -8J$ \\ 
$
\left[ \begin{array}{cc} 1 & -1 \\
                             	    -1  & 1 \\
\end{array} \right]$ &
$
\left[ \begin{array}{cc} -1 & 1 \\
                             	    1  & -1 \\
\end{array} \right]$ &
$
\left[ \begin{array}{cc} -1 & -1 \\
                             	    -1  & 1 \\
\end{array} \right]$ &
$
\left[ \begin{array}{cc} 1 & 1 \\
                             	    1  & 1 \\
\end{array} \right]$
\end{tabular}
\\
We then see that there is only three possible values for the energies, $E \in \{-8J, 0, 8J\}$ with corresponding multiplicity $\{2, 12, 2\}$. The partition function can then be calculated:
\begin{align*}
Z &= 2e^{-\beta (-8J)} + 12e^{-\beta \cdot 0} + 2 e^{-\beta \cdot 8J} = 2(e^{\beta \cdot 8J}  + e^{-\beta \cdot 8J}) + 12 \\
&= 4\cosh(8J\beta) + 12
\end{align*}
using that $\cosh(x) = \frac{1}{2}(e^{-x} + e^x)$.

Now that we have the partition function various expectation variables can be calculated. We start with the expectation value for the energy:
\begin{align*}
\langle E \rangle &= \frac{1}{Z} \sum_{i=1}^M E_ie^{-\beta E_i} = \frac{1}{Z} \big[ 2\cdot (-8J)e^{8J\beta} + 0 + 2\cdot 8Je^{-8J\beta}\big] = \frac{1}{Z} \big[-16Je^{8J\beta} + 16Je^{-8J\beta}\big] \\
&= -\frac{16J}{Z}\big[-e{-8J\beta} + e^{8J\beta}\big] = -\frac{32J}{Z}\sinh(8J\beta) = -\frac{8J\sinh(8J\beta)}{\cosh(8J\beta) + 3}
\end{align*}
using that $\sinh(x) = \frac{1}{2}(-e^{-x} + e^x)$. We will now calculate the heat capacity 
\[
C_v = \frac{1}{k_bT}\sigma_E^2 =  \frac{1}{k_bT}\big(\langle E^2 \rangle - \langle E \rangle^2\big)
\]
where
\begin{align*}
\langle E^2 \rangle &= \frac{1}{Z} \sum_{i=1}^M E_i^2 e^{-\beta E_i} = \frac{1}{Z} \big[ 2\cdot (-8J)^2 e^{8J\beta} + 0 + 2\cdot (8J)^2 e^{-8J\beta}\big] = \frac{128J^2}{Z} \big[ e^{8J\beta} + e^{-8J\beta}\big] \\
&= \frac{128J^2 \cdot 2\cosh(8J\beta)}{4\cosh(8J\beta) + 12} =  \frac{64J^2 \cosh(8J\beta)}{\cosh(8J\beta) + 3}
\end{align*}
so that 
\begin{align*}
C_v &= \frac{1}{k_bT}\bigg[ \frac{64J^2 \cosh(8J\beta)}{\cosh(8J\beta) + 3} - \bigg(  -\frac{8J\sinh(8J\beta)}{\cosh(8J\beta) + 3} \bigg)^2 \bigg] = \frac{1}{k_bT}\bigg[ \frac{64J^2 \cosh(8J\beta)}{\cosh(8J\beta) + 3} - \frac{64J^2 \sinh^2(8J\beta)}{(\cosh(8J\beta) + 3)^2} \bigg] \\
&= \frac{64J^2}{k_bT}\bigg[ \frac{ \cosh(8J\beta)(\cosh(8J\beta) + 3)-  \sinh^2(8J\beta)}{(\cosh(8J\beta) + 3)^2} \bigg] = \frac{64J^2}{k_bT}\bigg[ \frac{\cosh^2(8J\beta) + 3\cosh^2(8J\beta) - \sinh^2(8J\beta)}{(\cosh(8J\beta) + 3)^2} \bigg] \\
&= \frac{64J^2\beta}{T} \bigg[\frac{1 + 3\cosh(8J\beta)}{(\cosh(8J\beta) + 3)^2} \bigg]
\end{align*}

We define the magnetization $\mathcal{M}$ as the sum of all the spins $s$. It is easy to see that for the $2\times 2$ system there are only 5 possible values for the magnetization, $\mathcal{M} \in \{-4, -2, 0, 2, 4\}$.  The corresponding energy and multiplicity is given in table \ref{tab: multiplic}. 

\begin{tabular}{|c|c|c|c|}
\hline 
\# spins up & Multiplicy & Energy & Magnetization \\ 
\hline 
4 & 1 & -8J & 4 \\ 
\hline 
3 & 4 & 0 & 2 \\ 
\hline 
2 & 4 & 0 & 0 \\ 
\hline 
2 & 2 & 8J & 0 \\ 
\hline 
1 & 4 & 0 & -2 \\ 
\hline 
0 & 1 & -8J & -4 \\ 
\hline 
\end{tabular} \label{tab: multiplic}

The expectation value of the magnetization and the magnetization squared can be calculated as
\begin{align}
\langle \mathcal{M} \rangle &= \frac{1}{Z}\sum_{i=1}^M \mathcal{M}  e^{-\beta E_i} \label{eq: exp M} \\
\langle \mathcal{M}^2 \rangle &= \frac{1}{Z}\sum_{i=1}^M \mathcal{M}^2  e^{-\beta E_i} \nonumber
\end{align}
which we can use to calculate the suceptibility, the parameter of special interest: 
\[
\chi = \frac{1}{k_bT}\sigma_\mathcal{M} = \frac{1}{k_bT}\big( \langle \mathcal{M}^2 \rangle - \langle \mathcal{M} \rangle^2 \big).
\]
When looking at equation \ref{eq: exp M} and table \ref{tab: multiplic} we quicly see that the expectation value of the magnetization is zero for all temperatures. The microstates whith oposite magnetization have the same multiplicity so they cancel each other out. When running numerical calculations on large systems, for large $L$, the numerical value of $\langle \mathcal{M} \rangle$ will not reach zero unless the simulation is run for a (extremely) long time. This is because it takes a long time for the simulation to go through all the possible microstates after the simulation have reached a very probable state. But we want to have a reasonable measure of the magnetization and suceptibility for large systems and at the same time minimize the excecution time of the numerical simulation. To do this we choose to use the absolute value of the magnetization in the definition of the suceptibility. 

There are 3 possible values for the absolute value of the magnetization, $|\mathcal{M}| \in \{0, 2, 4\}$.  The corresponding energy and multiplicity is given in table \ref{tab: multiplic2}. 

\begin{tabular}{|c|c|c|}
\hline 
Absolute Magnetization & Energy &  Multiplicy\\ 
\hline 
4 & -8J & 2 \\ 
\hline 
2 & 0 & 8 \\ 
\hline 
0 & 0 & 4 \\ 
\hline 
0 & 8J & 2 \\ 
\hline 
\end{tabular} \label{tab: multiplic2}

We are then ready to calculate the expectation value of the absolute value of the magnetization by using the multiplicity for each energy and corresponding magnetization:
\begin{align}
\langle |\mathcal{M}| \rangle &= \frac{1}{Z}\sum_{i=1}^M |\mathcal{M}|  e^{-\beta E_i} = \frac{1}{Z} \bigg( 2\cdot 4e^{-\beta(-8J)} + 8\cdot 2e^0 + 0\cdot4e^0 + 0\cdot 2e^{-\beta 8J} \bigg)\nonumber \\
&= \frac{8}{Z}\big(e^{8J\beta} + 2\big) = \frac{8(e^{8J\beta} + 2)}{4\cosh(8j\beta) + 12} = \frac{2(e^{8J\beta}+ 2)}{\cosh(8j\beta) + 3}.
\label{eq: exp absM} 
\end{align}
We also obtain the expectation value of the square of the magnetization:
\begin{align*}
\langle \mathcal{M}^2 \rangle &= \frac{1}{Z}\sum_{i=1}^M \mathcal{M}^2 e^{-\beta E_i} = \frac{1}{Z}\big( 2\cdot 4^2 e^{-\beta (-8J)} + 8\cdot 2^2e^0 + 0 + 0 \big) = \frac{32}{Z}\big( e^{8J\beta} + 1 \big) \\
&= \frac{8\big(e^{8J\beta} + 1\big)}{\cosh(8J\beta) + 3}.
\end{align*}
And finally the suceptibility is given by
\begin{align*}
\chi &= \frac{1}{k_bT}\big( \langle \mathcal{M}^2 \rangle - \langle |\mathcal{M}| \rangle^2 \big) = \frac{1}{k_bT}\bigg[ \frac{8(e^{8J\beta} + 1)}{\cosh(8J\beta) + 3} - \bigg( \frac{2(e^{8J\beta}+ 2)}{\cosh(8j\beta) + 3} \bigg)^2 \bigg] \\
&= \frac{1}{k_bT}\bigg[ \frac{8(e^{8J\beta} + 1\big)(\cosh(8j\beta) + 3) - 4(e^{8J\beta}+ 2)^2}{(\cosh(8j\beta) + 3)^2} \bigg] \\
&= 4\beta \bigg[ \frac{2(e^{8J\beta} + 1\big)(\cosh(8j\beta) + 3) - (e^{8J\beta}+ 2)^2}{(\cosh(8j\beta) + 3)^2} \bigg] 
\end{align*}

\subsection{Units, scaled parameters}
We will now write the equations for the expectation values, the heat capacity and the susceptibility in terms of a scaled temperature
\begin{align*}
T' &= T \frac{k_b}{J} \\
\Rightarrow T &= T'\frac{J}{k_b}\\
\Rightarrow \beta &= \frac{1}{k_bT} = \frac{1}{k_b\cdot T'\frac{J}{k_b}} = \frac{1}{T'J}
\end{align*}
We then set $J=1$ so that:
\[
\beta = \frac{1}{T'}
\]
We will use this new temperature $T'$ in the numerical calculations. We also scale the heat capacity as:
\[
C_v' = \frac{C_v}{k_b} = \frac{64}{T'^2} \frac{1 + 3\cosh(8/T')}{(\cosh(8/T') + 3)^2} 
\]
In terms of $T'$ the expectation values and the susceptibility for the $2\times 2$ system can be written as:
\begin{align}
\langle E \rangle &= -\frac{8\sinh(8/T')}{\cosh(8/T') + 3} \\
\langle E^2 \rangle &=\frac{64J^2 \cosh(8/T')}{\cosh(8/T') + 3} \\
\langle |\mathcal{M}| \rangle &= \frac{2(e^{8/T'}+ 2)}{\cosh(8/T') + 3}\\
\langle \mathcal{M}^2 \rangle &= \frac{8\big(e^{8/T'} + 1\big)}{\cosh(8/T') + 3} \\
\chi &= \frac{4}{T'} \frac{2(e^{8/T'} + 1\big)(\cosh(8/T') + 3) - (e^{8/T'}+ 2)^2}{(\cosh(8/T') + 3)^2} 
\end{align}
These theoretical values will be used to test the implementation of the numerical method for the $L=2$ system. 

%theory mc methods/metropolis
%explaination of code

\section{Numerical methods}
bla bla theory and the programs
\subsection{Metropolis algorithm}

\begin{enumerate}
\item 
\end{enumerate}


\subsection{Coding dE  and w efficiently}
Finding expressions for the difference in the energy 





\section{"Experiment"}
\subsection{Testing the implementation og model and algorithm}

\subsection{Studying the number of Monte Carlo cycles}


\subsection{Studying .... close to the critical temp}


\subsection{Calculations to find $T_c$}


\section{Results and output}


\section{Discussion and experiences}





\newpage

\begin{enumerate}

\item[a)] Assume we have only two spins in each dimension, that is $L=2$.
Find the closed form expression for the partition function and the corresponding
expectations values for
for $E$, $|{\cal M}|$, the specific heat $C_V$ and the susceptibility $\chi$ 
as functions of  $T$ using periodic boundary conditions.



\item[b)] 
Write now a code for the Ising model which computes the mean energy 
$E$, mean magnetization 
$|{\cal M}|$, the specific heat $C_V$ and the susceptibility $\chi$ 
as functions of  $T$ using periodic boundary conditions for 
$L=2$ in the $x$ and $y$ directions. 
Compare your results with the expressions from a)
for  a  temperature $T=1.0$ (in units of $kT/J$). 

How many Monte Carlo cycles do you need in order to achieve a good agreeement?


\item[c)]
 
We choose now a square lattice with $L=20$ spins in the $x$ and $y$ directions. 

In [b) we did not study carefully how many Monte Carlo cycles were needed in order to reach the most likely state. Here
we want to perform a study of the time (here it corresponds to the number 
of Monte Carlo cycles) one needs before one reaches an equilibrium situation 
and can start computing various expectations values. Our 
first attempt is a rough and plain graphical
one, where we plot various expectations values as functions of the number of Monte Carlo cycles.

Choose first a temperature of $T=1.0$ (in units of $kT/J$) and study the 
mean energy and magnetisation (absolute value) as functions of the number of Monte Carlo cycles.
Use both an ordered (all spins pointing in one direction) and a random
spin orientation as starting configuration. 
How many Monte Carlo cycles do you need before you reach an equilibrium situation?
Repeat this analysis for $T=2.4$. 

Make also a plot of the total number of accepted configurations 
as function of the total number of Monte Carlo cycles. How does the number of
accepted configurations behave as function of temperature $T$?


\item[d)] Compute thereafter the probability 
$P(E)$ for the previous system with $L=20$ and the same temperatures.
You compute this probability by simply counting the number of times a 
given energy appears in your computation. Start the computation after 
the steady state situation has been reached.
Compare your results with the computed variance in energy 
$\sigma^2_E$ and discuss the behavior you observe. 
\end{enumerate}

Near $T_C$ we can characterize the behavior of many physical quantities
by a power law behavior.
As an example the mean magnetization is given by
\begin{equation}
  \langle {\cal M}(T) \rangle \sim \left(T-T_C\right)^{\beta},
\end{equation}
where $\beta=1/8$ is a so-called critical exponent. A similar relation
applies to the heat capacity 
\begin{equation}
  C_V(T) \sim \left|T_C-T\right|^{\alpha},
\end{equation}
and the susceptibility
\begin{equation}
  \chi(T) \sim \left|T_C-T\right|^{\gamma},
\end{equation}
with $\alpha = 0$ and $\gamma = 7/4$.
Another important quantity is the correlation length, which is expected
to be of the order of the lattice spacing for $T>> T_C$. Because the spins
become more and more correlated as $T$ approaches $T_C$, the correlation
length increases as we get closer to the critical temperature. The divergent
behavior of $\xi$ near $T_C$ 
is
\begin{equation}
  \xi(T) \sim \left|T_C-T\right|^{-\nu}.
  \label{eq:xi}
\end{equation}
A second-order phase transition is characterized by a
correlation length which spans the whole system.
Since we are always limited to a finite lattice, $\xi$ will
be proportional with the size of the lattice. 
Through so-called finite size scaling relations
it is possible to relate the behavior at finite lattices with the 
results for an infinitely large lattice.
The critical temperature scales then as
\begin{equation}
 T_C(L)-T_C(L=\infty) = aL^{-1/\nu},
 \label{eq:tc}
\end{equation}
with  $a$ a constant and  $\nu$ defined in Eq.~(\ref{eq:xi}).
We set $T=T_C$ and obtain a mean magnetisation
\begin{equation}
  \langle {\cal M}(T) \rangle \sim \left(T-T_C\right)^{\beta}
  \rightarrow L^{-\beta/\nu},
  \label{eq:scale1}
\end{equation}
a heat capacity
\begin{equation}
  C_V(T) \sim \left|T_C-T\right|^{-\gamma} \rightarrow L^{\alpha/\nu},
  \label{eq:scale2}
\end{equation}
and susceptibility
\begin{equation}
  \chi(T) \sim \left|T_C-T\right|^{-\alpha} \rightarrow L^{\gamma/\nu}.
  \label{eq:scale3}
\end{equation}

\begin{enumerate}
\item [e)]  We wish to study the behavior of the Ising model in two dimensions close to the 
critical temperature as a function of the lattice size $L\times L$.
Calculate the expectation values 
for $\langle E\rangle$ and $\langle |{\cal M}|\rangle$,
the specific heat $C_V$ and the susceptibility $\chi$
as functions  of $T$ for $L=20$, $L=40$, $L=60$ and $L=80$ for $T\in [2.0,2.4]$
with a step in temperature $\Delta T=0.05$ or smaller. 
Plot  $\langle E\rangle$, $\langle 1{\cal M}1\rangle$, $C_V$ and $\chi$ 
as functions of $T$. Can you see an indication of a phase transition?

Use the absolute value $\langle |{\cal M}|\rangle$ when you evaluate $\chi$.

You should parallelize the code using either OpenMP or MPI.
\item[f)]  Use Eq.~(\ref{eq:tc}) and the exact result
$\nu=1$ in order to estimate $T_C$ in the thermodynamic limit $L\rightarrow \infty$
using your simulations with $L=20$, $L=40$, $L=60$ and $L=80$
The exact result for the critical temperature (after Lars Onsager) is
$kT_C/J=2/ln(1+\sqrt{2})\approx 2.269$ with $\nu=1$.
\end{enumerate}

\section*{Background literature}
If you wish to read more about the Ising model and statistical physics here are two suggestions.
\begin{enumerate}
\item M.~Plischke and B.~Bergersen, Equilibrium Statistical Physics,
Prentice-Hall, see chapters 5 and 6.
\item M.~E.~J.~Newman and T.~Barkema, Monte Carlo methods in statistical physics, Oxford, see chapters 3 and 4.

\end{enumerate}
\end{document}






