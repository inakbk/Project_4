\documentclass[11pt,a4wide]{article}
\usepackage{verbatim}
\usepackage{listings}
\usepackage{graphicx}
\usepackage{a4wide}
\usepackage{color}
\usepackage{amsmath}
\usepackage{amssymb}
\usepackage[dvips]{epsfig}
\usepackage[T1]{fontenc}
\usepackage{cite} % [2,3,4] --> [2--4]
\usepackage{shadow}
\usepackage{hyperref}
\usepackage{caption}

\setcounter{tocdepth}{2}

\usepackage{subcaption}

\lstset{language=c++}
\lstset{alsolanguage=[90]Fortran}
\lstset{basicstyle=\small}
\lstset{backgroundcolor=\color{white}}
\lstset{frame=single}
\lstset{stringstyle=\ttfamily}
\lstset{keywordstyle=\color{red}\bfseries}
\lstset{commentstyle=\itshape\color{blue}}
\lstset{showspaces=false}
\lstset{showstringspaces=false}
\lstset{showtabs=false}
\lstset{breaklines}

%lager heftig forside:
\newcommand*{\titleAT}{\begingroup % Create the command for including the title page in the document
\newlength{\drop} % Command for generating a specific amount of whitespace
\drop=0.1\textheight % Define the command as 10% of the total text height

\rule{\textwidth}{1pt}\par % Thick horizontal line
\vspace{2pt}\vspace{-\baselineskip} % Whitespace between lines
\rule{\textwidth}{0.4pt}\par % Thin horizontal line

\vspace{0.5\drop} % Whitespace between the top lines and title
\centering % Center all text
\textcolor{black}{ % Red font color
{\Huge Studying second order phase transitions in the Ising model in two dimentions with the metropolis algorithm}\\[0.75\baselineskip] % Title line 1
%{\Large Tema:}\\[0.75\baselineskip] % Title line 2
%{\Huge Lydmåling og hørselstesting} % Title line 3
} 

\vspace{0.25\drop} % Whitespace between the title and short horizontal line
\rule{0.3\textwidth}{0.4pt}\par % Short horizontal line under the title
\vspace{0.25\drop} % Whitespace between the thin horizontal line and the author name

{\Large \textsc{Project 4, FYS-3150\\[0.75\baselineskip] \normalsize{Ina K. B. Kullmann, candidate nr: 20}
}}\par % Author name

%\vfill % Whitespace between the author name and publisher text

\vspace{0.25\drop} % Whitespace between the title and short horizontal line
\rule{0.3\textwidth}{0.4pt}\par % Short horizontal line under the title
\vspace{0.25\drop} % Whitespace between the thin horizontal line and the author name

\begin{abstract}
The aim of this project is to numerically find the critical temperature for the two dimentional Ising model by using the metropolis algorithm. We will first test the implementation of the algorithm carefully, first by comparing with theoretical values calculated for a small system. Then we will see if the algorithm behaves as expected according to our physical intuition for a larger system.

When we have found a estimate for the critical temperature we will compare it to Lars Onsagers analytical result.

\end{abstract}
\vspace*{0.25\drop} % Whitespace under the publisher text

\begin{center}
{ \scriptsize \noindent All source codes can be found at: \texttt{https://github.com/inakbk/Project\_4}. }
\end{center}

\rule{\textwidth}{0.4pt}\par % Thin horizontal line
\vspace{2pt}\vspace{-\baselineskip} % Whitespace between lines
\rule{\textwidth}{1pt}\par % Thick horizontal line

\endgroup}
%kode slutt for heftig forside


\begin{document}
%\maketitle
\titleAT % This command includes the title page


\newpage
\tableofcontents
\newpage

\section{Introduction}
A second order phase transition at the critical temperature $T_c$ is described by a macroscopic change to a system. At the critical temperature some termodynamic values approaches zero with an infinite slope and 'disappeares'. Other variables are discontinous or diverge at the critical temperature in the termodynamic limit. 

In this paper we will study such a phase transistion by using the Ising model and the metropolis algorithm to simulate results. The approach used in this paper is also applicabile for other physical models and therefore very useful.

\section{Theory}

\subsection{The Ising model in two dimentions}
The Ising model is a simple model for ferromagnetism in statistical physics. The model consists of magnetic spins that are allowed to interact with its neigbours. The magnetic dipole moments are allowed to have two values, 1 and -1.

In this project we will study the ising model i two dimentions wich allows the identification of a second order phase transition. We have chosen to set the external magnetic field to zero troughout this paper.

The Ising model gives that the energy for the whole system can be expressed as
\begin{equation}
  E=-J\sum_{<kl>}^{N}s_ks_l
\end{equation}
where the value of the spins are  $s_k=\pm 1$ and $N$ is the total number of spins. The variable $J$ is a coupling constant expressing the strength of the interaction between neighboring spins. We will divide all variables by the number of spins to obtain all results per spin. 

The symbol $<kl>$ indicates that we sum over nearest neighbors only. We will assume that we have a ferromagnetic ordering, viz $J> 0$ and use periodic boundary conditions. We choose to use a square lattice of $L$ spins in x and y direction. The total number of spins is then $N = L^2$ and the total number of microstates, the different configurations the spins can have is given by $M=2^N$

The partition function for such a system is then given by:
\[
Z = \sum_{i=1}^M e^{-\beta E_i}
\]

\subsection{Obtaining analytical values for the simple $L=2$ system} \label{sec:theory_L2}
First we assume that there is only two spins in each dimention (x and y), we set $L=2 \Rightarrow N = 2\cdot 2 = 4\Rightarrow M= 2^4 = 16$. So there is 16 possible energies of the spin system when $L=2$. So to calculate the partition function we have to find the energies:
\[
Z = e^{-\beta E_1} + ... e^{-\beta E_{16}}
\]

The spin system can be visualized as 16 matrices on the form:
$\left[ \begin{array}{cc} s(0,0) & s(0,1) \\
                             	    s(1,0)  & s(1,1) \\
\end{array} \right]$
with corresponding energies. 

Periodic boundaryconditions give that every spin has a neighbour. The neighbours of the spin $s(1,1)$ is  $s(0,1)$ twice (above and below) and $s(1,0)$ twice (above and below). To find the energy we have to sum over the nearest neighbours for all the spins in the system. The product $s_ls_k$ of two neighbours should only be calculated once. This is solved if the index of $s_l$ is fixed while $s_k$ has one higher index in each direction at a time:

\begin{align}
E &= -J\bigg( s(0,0)\cdot \big[s(1,0) + s(0,1) \big] + s(0,1)\cdot\big[s(1,1) + s(0,0) \big] \nonumber   \\ 
 &+ s(1,0)\cdot\big[s(0,0) + s(1,1) \big] + s(1,1)\cdot\big[s(0,1) + s(1,0) \big] \bigg)
 \label{eq: energy spins}
\end{align}

The 16 states with corresponding energies calculated with equation \ref{eq: energy spins} is given in table \ref{tab: energies}.

\begin{table}
\centering
\caption{All possible spin combinations and energies for the $L=2$ system.}
\begin{tabular}{cccc}
$E_1 = 0$ & $E_2 = 0$ & $E_3 = 0$& $E_4 = 0$ \\ 
$\left[ \begin{array}{cc} 1 & 1 \\
                             	    -1  & -1 \\
\end{array} \right]$ & 
$
\left[ \begin{array}{cc} 1 & -1 \\
                             	    1  & -1 \\
\end{array} \right]$ &
$
\left[ \begin{array}{cc} -1 & 1 \\
                             	    -1  & 1 \\
\end{array} \right]$ & 
$
\left[ \begin{array}{cc} -1 & -1 \\
                             	    1  & 1 \\
\end{array} \right] $ \\
$E_5 = 0$ & $E_6 = 0$ & $E_7 = 0$& $E_8 = 0$ \\ 
$
\left[ \begin{array}{cc} 1 & 1 \\
                             	    1  & -1 \\
\end{array} \right]$ &
$
\left[ \begin{array}{cc} 1 & 1 \\
                             	    -1  & 1 \\
\end{array} \right]$ &
$
\left[ \begin{array}{cc} 1 & -1 \\
                             	    1  & 1 \\
\end{array} \right]$ &
$
\left[ \begin{array}{cc} -1 & 1 \\
                             	    1  & 1 \\
\end{array} \right]$ \\
$E_9 = 0$ & $E_{10} = 0$ & $E_{11} = 0$& $E_{12} = 0$ \\ 
$
\left[ \begin{array}{cc} -1 & -1 \\
                             	    -1  & 1 \\
\end{array} \right]$ &
$
\left[ \begin{array}{cc} -1 & -1 \\
                             	    1  & -1 \\
\end{array} \right]$ &
$
\left[ \begin{array}{cc} 1 & -1 \\
                             	    -1  & -1 \\
\end{array} \right]$ &
$
\left[ \begin{array}{cc} -1 & 1 \\
                             	    -1  & -1 \\
\end{array} \right]$ \\
$E_{13} = 8J$ & $E_{14} = 8J$ & $E_{15} = -8J$& $E_{16} = -8J$ \\ 
$
\left[ \begin{array}{cc} 1 & -1 \\
                             	    -1  & 1 \\
\end{array} \right]$ &
$
\left[ \begin{array}{cc} -1 & 1 \\
                             	    1  & -1 \\
\end{array} \right]$ &
$
\left[ \begin{array}{cc} -1 & -1 \\
                             	    -1  & 1 \\
\end{array} \right]$ &
$
\left[ \begin{array}{cc} 1 & 1 \\
                             	    1  & 1 \\
\end{array} \right]$
\end{tabular}
\label{tab: energies}
\end{table}

Looking at table \ref{tab: energies} we see that there is only three possible values for the energies, $E \in \{-8J, 0, 8J\}$ with corresponding multiplicity $\{2, 12, 2\}$. The partition function can then be calculated:
\begin{align*}
Z &= 2e^{-\beta (-8J)} + 12e^{-\beta \cdot 0} + 2 e^{-\beta \cdot 8J} = 2(e^{\beta \cdot 8J}  + e^{-\beta \cdot 8J}) + 12 \\
&= 4\cosh(8J\beta) + 12
\end{align*}
using that $\cosh(x) = \frac{1}{2}(e^{-x} + e^x)$.

Now that we have the partition function various expectation variables can be calculated. We start with the expectation value for the energy:
\begin{align*}
\langle E \rangle &= \frac{1}{Z} \sum_{i=1}^M E_ie^{-\beta E_i} = \frac{1}{Z} \big[ 2\cdot (-8J)e^{8J\beta} + 0 + 2\cdot 8Je^{-8J\beta}\big] = \frac{1}{Z} \big[-16Je^{8J\beta} + 16Je^{-8J\beta}\big] \\
&= -\frac{16J}{Z}\big[-e{-8J\beta} + e^{8J\beta}\big] = -\frac{32J}{Z}\sinh(8J\beta) = -\frac{8J\sinh(8J\beta)}{\cosh(8J\beta) + 3}
\end{align*}
using that $\sinh(x) = \frac{1}{2}(-e^{-x} + e^x)$. We will now calculate the heat capacity 
\[
C_v = \frac{1}{k_bT}\sigma_E^2 =  \frac{1}{k_bT}\big(\langle E^2 \rangle - \langle E \rangle^2\big)
\]
where
\begin{align*}
\langle E^2 \rangle &= \frac{1}{Z} \sum_{i=1}^M E_i^2 e^{-\beta E_i} = \frac{1}{Z} \big[ 2\cdot (-8J)^2 e^{8J\beta} + 0 + 2\cdot (8J)^2 e^{-8J\beta}\big] = \frac{128J^2}{Z} \big[ e^{8J\beta} + e^{-8J\beta}\big] \\
&= \frac{128J^2 \cdot 2\cosh(8J\beta)}{4\cosh(8J\beta) + 12} =  \frac{64J^2 \cosh(8J\beta)}{\cosh(8J\beta) + 3}
\end{align*}
so that 
\begin{align*}
C_v &= \frac{1}{k_bT}\bigg[ \frac{64J^2 \cosh(8J\beta)}{\cosh(8J\beta) + 3} - \bigg(  -\frac{8J\sinh(8J\beta)}{\cosh(8J\beta) + 3} \bigg)^2 \bigg] = \frac{1}{k_bT}\bigg[ \frac{64J^2 \cosh(8J\beta)}{\cosh(8J\beta) + 3} - \frac{64J^2 \sinh^2(8J\beta)}{(\cosh(8J\beta) + 3)^2} \bigg] \\
&= \frac{64J^2}{k_bT}\bigg[ \frac{ \cosh(8J\beta)(\cosh(8J\beta) + 3)-  \sinh^2(8J\beta)}{(\cosh(8J\beta) + 3)^2} \bigg] = \frac{64J^2}{k_bT}\bigg[ \frac{\cosh^2(8J\beta) + 3\cosh^2(8J\beta) - \sinh^2(8J\beta)}{(\cosh(8J\beta) + 3)^2} \bigg] \\
&= \frac{64J^2\beta}{T} \bigg[\frac{1 + 3\cosh(8J\beta)}{(\cosh(8J\beta) + 3)^2} \bigg]
\end{align*}

We define the magnetization $\mathcal{M}$ as the sum of all the spins $s$. It is easy to see that for the $2\times 2$ system there are only 5 possible values for the magnetization, $\mathcal{M} \in \{-4, -2, 0, 2, 4\}$.  The corresponding energy and multiplicity is given in table \ref{tab: multiplic}. 

\begin{table}
\centering
\caption{Magnetization with the corresponding energy and multiplicity for the $L=2$ system.}
\begin{tabular}{|c|c|c|c|}
\hline 
\# spins up & Multiplicy & Energy & Magnetization \\ 
\hline 
4 & 1 & -8J & 4 \\ 
\hline 
3 & 4 & 0 & 2 \\ 
\hline 
2 & 4 & 0 & 0 \\ 
\hline 
2 & 2 & 8J & 0 \\ 
\hline 
1 & 4 & 0 & -2 \\ 
\hline 
0 & 1 & -8J & -4 \\ 
\hline 
\end{tabular} 
\label{tab: multiplic}
\end{table}
The expectation value of the magnetization and the magnetization squared can be calculated as
\begin{align}
\langle \mathcal{M} \rangle &= \frac{1}{Z}\sum_{i=1}^M \mathcal{M}  e^{-\beta E_i} \label{eq: exp M} \\
\langle \mathcal{M}^2 \rangle &= \frac{1}{Z}\sum_{i=1}^M \mathcal{M}^2  e^{-\beta E_i} \nonumber
\end{align}
which we can use to calculate the suceptibility, the parameter of special interest: 
\[
\chi = \frac{1}{k_bT}\sigma_\mathcal{M} = \frac{1}{k_bT}\big( \langle \mathcal{M}^2 \rangle - \langle \mathcal{M} \rangle^2 \big).
\]
When looking at equation \ref{eq: exp M} and table \ref{tab: multiplic} we quicly see that the expectation value of the magnetization is zero for all temperatures. The microstates whith oposite magnetization have the same multiplicity so they cancel each other out. When running numerical calculations on large systems, for large $L$, the numerical value of $\langle \mathcal{M} \rangle$ will not reach zero unless the simulation is run for a (extremely) long time. This is because it takes a long time for the simulation to go through all the possible microstates after the simulation have reached a very probable state. But we want to have a reasonable measure of the magnetization and suceptibility for large systems and at the same time minimize the excecution time of the numerical simulation. To do this we choose to use the absolute value of the magnetization in the definition of the suceptibility. 

There are 3 possible values for the absolute value of the magnetization, $|\mathcal{M}| \in \{0, 2, 4\}$.  The corresponding energy and multiplicity is given in table \ref{tab: multiplic2}. 

\begin{table}
\centering
\caption{Absolute magnetization with the corresponding energy and multiplicity for the $L=2$ system.}
\begin{tabular}{|c|c|c|}
\hline 
Absolute Magnetization & Energy &  Multiplicy\\ 
\hline 
4 & -8J & 2 \\ 
\hline 
2 & 0 & 8 \\ 
\hline 
0 & 0 & 4 \\ 
\hline 
0 & 8J & 2 \\ 
\hline 
\end{tabular} 
\label{tab: multiplic2}
\end{table}

We are then ready to calculate the expectation value of the absolute value of the magnetization by using the multiplicity for each energy and corresponding magnetization:
\begin{align}
\langle |\mathcal{M}| \rangle &= \frac{1}{Z}\sum_{i=1}^M |\mathcal{M}|  e^{-\beta E_i} = \frac{1}{Z} \bigg( 2\cdot 4e^{-\beta(-8J)} + 8\cdot 2e^0 + 0\cdot4e^0 + 0\cdot 2e^{-\beta 8J} \bigg)\nonumber \\
&= \frac{8}{Z}\big(e^{8J\beta} + 2\big) = \frac{8(e^{8J\beta} + 2)}{4\cosh(8j\beta) + 12} = \frac{2(e^{8J\beta}+ 2)}{\cosh(8j\beta) + 3}.
\label{eq: exp absM} 
\end{align}
We also obtain the expectation value of the square of the magnetization:
\begin{align*}
\langle \mathcal{M}^2 \rangle &= \frac{1}{Z}\sum_{i=1}^M \mathcal{M}^2 e^{-\beta E_i} = \frac{1}{Z}\big( 2\cdot 4^2 e^{-\beta (-8J)} + 8\cdot 2^2e^0 + 0 + 0 \big) = \frac{32}{Z}\big( e^{8J\beta} + 1 \big) \\
&= \frac{8\big(e^{8J\beta} + 1\big)}{\cosh(8J\beta) + 3}.
\end{align*}
And finally the suceptibility is given by
\begin{align*}
\chi &= \frac{1}{k_bT}\big( \langle \mathcal{M}^2 \rangle - \langle |\mathcal{M}| \rangle^2 \big) = \frac{1}{k_bT}\bigg[ \frac{8(e^{8J\beta} + 1)}{\cosh(8J\beta) + 3} - \bigg( \frac{2(e^{8J\beta}+ 2)}{\cosh(8j\beta) + 3} \bigg)^2 \bigg] \\
&= \frac{1}{k_bT}\bigg[ \frac{8(e^{8J\beta} + 1\big)(\cosh(8j\beta) + 3) - 4(e^{8J\beta}+ 2)^2}{(\cosh(8j\beta) + 3)^2} \bigg] \\
&= 4\beta \bigg[ \frac{2(e^{8J\beta} + 1\big)(\cosh(8j\beta) + 3) - (e^{8J\beta}+ 2)^2}{(\cosh(8j\beta) + 3)^2} \bigg] 
\end{align*}

\subsection{Units and scaled parameters}
We will now write the equations for the expectation values, the heat capacity and the susceptibility in terms of a scaled temperature
\begin{align*}
T' &= T \frac{k_b}{J} \\
\Rightarrow T &= T'\frac{J}{k_b}\\
\Rightarrow \beta &= \frac{1}{k_bT} = \frac{1}{k_b\cdot T'\frac{J}{k_b}} = \frac{1}{T'J}
\end{align*}
We then set $J=1$ so that:
\[
\beta = \frac{1}{T'}
\]
We will use this new temperature $T'$ in the numerical calculations. We also scale the heat capacity as:
\[
C_v' = \frac{C_v}{k_b} = \frac{64}{T'^2} \frac{1 + 3\cosh(8/T')}{(\cosh(8/T') + 3)^2} 
\]
In terms of $T'$ the expectation values and the susceptibility for the $2\times 2$ system can be written as:
\begin{align}
\langle E \rangle &= -\frac{8\sinh(8/T')}{\cosh(8/T') + 3} \\
\langle E^2 \rangle &=\frac{64J^2 \cosh(8/T')}{\cosh(8/T') + 3} \\
\langle |\mathcal{M}| \rangle &= \frac{2(e^{8/T'}+ 2)}{\cosh(8/T') + 3}\\
\langle \mathcal{M}^2 \rangle &= \frac{8\big(e^{8/T'} + 1\big)}{\cosh(8/T') + 3} \\
\chi &= \frac{4}{T'} \frac{2(e^{8/T'} + 1\big)(\cosh(8/T') + 3) - (e^{8/T'}+ 2)^2}{(\cosh(8/T') + 3)^2} 
\end{align}
These theoretical values will be used to test the implementation of the numerical method for the $L=2$ system. 

\subsection{Second order phase transistions}
Near $T_C$, where the phase transition happens, we can characterize the behavior of many physical quantities by a power law behavior.
As an example the mean magnetization is given by
\begin{equation}
  \langle {\cal M}(T) \rangle \sim \left(T-T_C\right)^{\beta},
\end{equation}
where $\beta=1/8$ is a so-called critical exponent. A similar relation
applies to the heat capacity 
\begin{equation} 
  C_V(T) \sim \left|T_C-T\right|^{\alpha},
\end{equation}
and the susceptibility
\begin{equation}
  \chi(T) \sim \left|T_C-T\right|^{\gamma},
\end{equation}
with $\alpha = 0$ and $\gamma = 7/4$.
Another important quantity is the correlation length, which is expected
to be of the order of the lattice spacing for $T>> T_C$. Because the spins
become more and more correlated as $T$ approaches $T_C$, the correlation
length increases as we get closer to the critical temperature. The divergent
behavior of $\xi$ near $T_C$ 
is
\begin{equation}
  \xi(T) \sim \left|T_C-T\right|^{-\nu}.
  \label{eq:xi}
\end{equation}
A second-order phase transition is characterized by a
correlation length which spans the whole system.
Since we are always limited to a finite lattice, $\xi$ will
be proportional with the size of the lattice. 
Through so-called finite size scaling relations
it is possible to relate the behavior at finite lattices with the 
results for an infinitely large lattice.
The critical temperature scales then as
\begin{equation}
 T_C(L)-T_C(L=\infty) = aL^{-1/\nu},
 \label{eq:tc}
\end{equation}
with  $a$ a constant and  $\nu=1$ defined in Eq.~(\ref{eq:xi}). This is the equation we will use to estimate the critical temperature with the numerical results and compare with Lars Onsagers theoretical results $T_c \approx 2.269$. 

Let us first rewrite the equation \ref{eq:tc} and using $T_C(L=\infty)x$ for the variable that we wish to estimate and using two different values of $L$. We then combine
\[
 T_C(L)-x = a/L\\
\]
with
\[
 T_C(L^*)-x = a/L^*\\
\]
and obtain
\[
x = \frac{T_c(L^*)L^* - T_c(L)L}{L^* - L}
\]

\section{Numerical methods}
\subsection{Metropolis algorithm}
We imagine to start out with an initial spin state that is random or ordered. We will then use the following approach to simulate the system develop in time:
\begin{enumerate}
\item The system are in a spin state with energy E
\item Create a trial state with the trial energy $E_t$ by flipping one spin
\item Compute $\Delta E = E_t - E$
\begin{itemize}
\item If $\Delta E \leq 0$
\begin{itemize}
\item Accept the trial state as the new state 
\end{itemize}
\item If $\Delta E >0$ calculate $w = e^{-\beta \Delta E}$ and create a random number $r$
\begin{itemize}
\item If $r\leq w$ (the metropolis test): Accept the trial state as the new state 
\item Else discard the trial state (do nothing with the state)
\end{itemize}
\end{itemize}
\item Do 1.-3. $L\times L$ times (one MC cycle)
\item Update mean values 
\item Reapeat 1.-6. untill enough statistics is sampled, the desired number of MC cycles
\end{enumerate}

One Monte Carlo cycle corresponds to one loop that have the possibility to flip all the spins in the lattice. 

The algorithm can also be expressed as a Psedocode:
\begin{lstlisting}
for numberOfTemperatures:
{
	//create initial state here
	//calculate initial energy and magnetization here
	for numberoOfMCcycles:
	{
		for numberOfspins:
		{
			//do one spin flip here
		}
		//update mean values here
		//print mean values to file here
	}	
	//or print values to file here
}

//then read data from file and plot in python
\end{lstlisting}

This is the set up that is used in this project. Some of the versions of the algorithm the temperature loop is left to the python code, which also compiles and runs the cpp code. This is one way of paralellizing the code. In the final version of the program we have paralellized the code using \texttt{openMPI}. 

We will also implement the algorithm so that we discard all contributions to the expectation valsues before the system have reached equilibrium. To do this we have to test the algorithm and study how many Monte Carlo cycles the system uses to reach the equilibrium state. 

The folder structure are as follows:
\begin{itemize}
\item Folders used for testing:
\begin{itemize}
\item \texttt{test\_ising\_metropolis} for the $L=2$ system
\item \texttt{equilibrium\_ising\_metropolis} to study the equlibrium time and the number of accepted cycles
\item \texttt{probability\_ising\_metropolist} to study the probability for a given energy and temperature
\end{itemize}
\item Final folder to find the critical temperature $T_c$:
\begin{itemize}
\item \texttt{ising\_metropolis\_MPI}
\end{itemize}
\item Folder for this report:
\begin{itemize}
\item \texttt{report4}
\end{itemize}
\end{itemize}
where all folders are found at \texttt{https://github.com/inakbk/Project\_4}.

\subsection{Coding $\Delta E$, $\Delta \mathcal{M}$  and w efficiently}
We wish to run the simulations for large numbers of Monte Carlo cycles $~10^5$. It is therefore very important to code the calculation of $\Delta E$ efficiently because it will be calculated $N$ times each Monte Carlo cycle. 

The difference in the energy can be expressed as:
\begin{align*}
\Delta E = E_2 - E_1 = -J\sum_{<kl>}^N s_{k,2}s_{l,2} + J\sum_{<kl>}^N s_{k,1}s_{l,1}
\end{align*}
where the $l$ index is the spin that was flipped and $k$ the neighbours of $l$.  We only flip one spin at a time, so the neighbours of $s_l$ are unchanged meaning that $s_{k,2} = s_{k,1} = s_k$. The difference in energy can then be written as:
\[
\Delta E = J\sum_{<kl>}^N s_ks_{l,1} - J\sum_{<kl>}^N s_ks_{l,2} = J\sum_{<kl>}^N s_k(s_{l,1} - s_{l,2} )
\]
The spin can only take two values, so $s_{l,1}$ is either 1 or -1. If $s_{l,1}=1$ then after the flip $s_{l,2} = -1$ and $s_{l,1} - s_{l,2} = 1 - (-1) = 2$. If $s_{l,1}=-1$ then after the flip $s_{l,2} = 1$ and $s_{l,1} - s_{l,2} = -1 - 1 = -2$. We can therefore write $s_{l,1} - s_{l,2} = 2s_{l,1}$ and then we obtain:
\[
\Delta E = J\sum_{<kl>}^N s_k\cdot 2s_{l,1} = 2Js_{l,1}\sum_{<k>}^N s_k
\]

Similarly for the magnetization:
\begin{align*}
\Delta \mathcal{M} &= \mathcal{M}_2 - \mathcal{M}_1 = \sum_i^N s_{i,2} - \sum_i^N s_{i,1} = \sum_i^N (s_{i,2} -  s_{i,1}) = s_{l,2} - s_{l,1} = -2s_{l,1}\\
\Rightarrow \mathcal{M}_2 &= \mathcal{M}_1 - 2s_{l,1}
\end{align*}

This way of calculating the difference in energy and magnetization makes the algorithm for flipping one spin run faster than orther solutions. We also want to avoid to calculate the exponential of $\Delta E $, the value of $w=e^{-\beta\Delta E}$ for every time the trial energy is higher than the current energy. We can avoid this by noticing that there are only 5 possible values $\Delta E$ can take when we only flip one spin at a time (in the two dimentional case) $\Delta E \in \{-8, -4, 0, 4, 8\}$. So the energy minimum have to change in steps of 4, and maximum in steps of 8 when we flip one spin at a time. Then there are also only 5 possible values of $w$ for a given temperature. We can exploit this further by realizing that we only use $w$ when $\Delta E > 0 \Rightarrow \Delta E \in \{4, 8\}$. For a given temperature we then calculate the two values of $w$ once and pass it on to the metropolis test which uses the value corresponding to $\Delta E$. 

With this methods implemented in the algorithms the execution time is much faster. 

\section{Implementation and results}
\subsection{Testing the implementation of the model and algorithm}
First we will compare the results for the metropolis algorithm with the theoretical values for the $L=2$ case calculated in section \ref{sec:theory_L2}. All calculations and results in this report is per spin, meaning that we have divided the values with the total number of spins $N=L^2$. 

In figure \ref{fig:theory E} we see the expectation value of the energy (top), energy squared (middle) and the heat capacity (bottom) as a function of the temperature. The simulation was run for $5\cdot10^5$ Monte Carlo Cycles. The simulation was started with an initial state with all spins up. In figure 

\begin{figure}[htp]
\centering
\includegraphics[width=0.8\textwidth]{figures_L2/theoryE.png}
\caption{Expectation value of the energy (top), energy squared (middle) and the heat capacity (bottom) per spin as a function of the temperature. The numerical calculation was run for $5\cdot10^5$ Monte Carlo Cycles with a initial state of all spins up (\texttt{initial\_state=1}) with $L=2$. The numerical (blue line) and the teoretical (green line) values are plotted together.}
\label{fig:theory E}
\end{figure}

\begin{figure}[htp]
\centering
\includegraphics[width=0.8\textwidth]{figures_L2/theoryM.png}
\caption{ Expectation value of the absolute magnetization (top), magnetization squared (middle) and the susceptibility (bottom) per spin as a function of the temperature. The numerical calculation was run for $5\cdot10^5$ Monte Carlo Cycles with a initial state of all spins up (\texttt{initial\_state=1}) with $L=2$. The numerical (blue line) and the teoretical (green line) values are plotted together.}
\label{fig:theory M}
\end{figure}

The numerical values agree very well with the theoretical values. In figure \ref{fig: error E} we see the absolute error in the Expectation value of the energy (top), energy squared (middle) and the heat capacity (bottom) per spin as a function of Monte Carlo cycles. In figure \ref{fig: error M} we see the absolute error in the expectation value of the absolute magnetization (top), magnetization squared (middle) and the susceptibility (bottom) as a function of Monte Carlo Cycles. The temperature in both plot \ref{fig: error E} and \ref{fig: error M} was held at $T=1$.

The spin system with $L=2$ is very small so it takes very few MC cycles to reach the equlilibrium state, which for low temperatures are close to the lowes energy state. I does therefore not matter which initial state the system start out with. The error in the numerical calculcations will not depend on the initial state, but on the amount of statistics, how many MC cycles. In the simulation run in figure \ref{fig: error E} we have used a initial state of all spins up (\texttt{initial\_state=1}) which is the state with the lowest energy.  

In the figures \ref{fig: error E} and \ref{fig: error M} we see that the error falls quicly and stabilizes after $~2\cdot 10^5$ MC cycles. The heat capacity has the largest error with absolute error below 0.01 after stabilizing. Most of the error comes from the expectation value of the energy squared while the error in the expectation value of the energy is much smaller, $~0.002$. The error in the suceptibility falls below 0.002, smallest error is for the the expectation value of the absolute magnetization of less than 0.0005.

From looking at figure \ref{fig: error E} and \ref{fig: error M} it seems reasonable to choose the number of MC cycles to be more than $2\cdot10^5$.

\begin{figure}[htp]
\centering
\includegraphics[width=0.8\textwidth]{figures_L2/error_E_MC.png}
\caption{Absolute error in the expectation value of the energy (top), energy squared (middle) and the heat capacity (bottom) a function of Monte Carlo Cycles with a initial state of all spins up (\texttt{initial\_state=1}) with $L=2$. The temperature was held at $T=1$. }
\label{fig: error E}
\end{figure}

\begin{figure}[htp]
\centering
\includegraphics[width=0.8\textwidth]{figures_L2/error_M_MC.png}
\caption{Absolute error in the expectation value of the absolute magnetization (top), magnetization squared (middle) and the susceptibility (bottom) as a function of Monte Carlo Cycles with a initial state of all spins up (\texttt{initial0.8\_state=1}) with $L=2$. The temperature was held at $T=1$. }
\label{fig: error M}
\end{figure}

\subsection{Equilibrium time}
We will now look at a larger system with $L=20$ and study the equilibrium time, how many Monte Carlo cycles the system uses to reach the stable equilibrium state. The calculations of mean values should starte after the equilibrium state is reached so that the statistics are not 'ruined' by the path from the initial state to the equilibrium state. We will therefore try to decide the equilibrium time. We will look at two temperatures, $T\in \{1.0, 2.4\}$ and start in both a random and ordered initial state. For the ordered state we will have all spins up. 

In figure \ref{fig: equilibrium T1} we see the expectation value of the energy per spin as a function of Monte Carlo cycles at the top and the expectation value of the absolute magnetization per spin as a function of Monte Carlo cycles on the bottom. The plots to the left are with the initial state ordered and we can clearly see that the system stabelazies around a value both for the mean energy and mean absolute magnetization. 

The plots to the right in figure \ref{fig: equilibrium T1} started with a random initial state and have a different unexpected behaviour. We see that the expectation values have a 'hump' before the stabilizing. The same strange behaviour can be observed in figure \ref{fig: equilibrium T2} which is the same plot, but for a higher temperature $T=2.4$. We see that the plots with random initial state are very smooth and do not have any statistically behaviour. 

\begin{figure}[htp]
\centering
\includegraphics[width=0.8\textwidth]{equilibrium/equilibrium_T1_MC.png}
\caption{Top: Expectation value of the energy per spin as a function of Monte Carlo cycles. Bottom:  Expectation value of the absolute magnetization per spin as a function of Monte Carlo cycles. To the left the initial state is ordered and to the right the initial state is random. The size of the lattice is $L=20$ and the temperature is $T=1$.}
\label{fig: equilibrium T1}
\end{figure}

\begin{figure}[htp]
\centering
\includegraphics[width=0.8\textwidth]{equilibrium/equilibrium_T2_MC.png}
\caption{Top: Expectation value of the energy per spin as a function of Monte Carlo cycles. Bottom:  Expectation value of the absolute magnetization per spin as a function of Monte Carlo cycles. To the left the initial state is ordered and to the right the initial state is random. The size of the lattice is $L=20$ and the temperature is $T=2.4$.}
\label{fig: equilibrium T2}
\end{figure}

In figure \ref{fig: accepted T1} we see the accepted configurations per spin as a function of MC cycles for both ordered and random initial state for the temperature $T=1$. We see that the number of accepted configurations per spin grows as number of MC cycles increase and then flats out. For the random initial state we see the oposite behaviour, the number of accepted configurations decrease as a function of MC cycles. We would expect that for a low temperature as $T=1$ the equilibrium state are closer to the orderes state than the random state. It is natural that for the random initial state the number of accepted configurations are very large in the start as the system is 'trying' to reach the equilibrium state, and when the equilibrium state is found the number of accepted states flats out. 

The corresponding plot for $T=2.4$ is shown in figure \ref{fig:  accepted T2}. We see that the number of accepted counfigurations flats out for the ordered and random initial state here too. But the Random initial state have a very sooth slope, which the low temperature case did not have. 


\begin{figure}[htp]
\includegraphics[width=0.8\textwidth]{equilibrium/accepted_T1_MC.png}
\caption{Accepted configurations per spin as a function of MC cycles. Top: Ordered initial state (all spinst up). Bottom: Random initial state. The temperature is $T=1$.}
\centering
\label{fig: accepted T1}
\end{figure}

\begin{figure}[htp]
\includegraphics[width=0.8\textwidth]{equilibrium/accepted_T2_MC.png}
\caption{Accepted configurations per spin as a function of MC cycles. Top: Ordered initial state (all spinst up). Bottom: Random initial state. The temperature is $T=2.4$.}
\centering
\label{fig:  accepted T2}
\end{figure}



In this subsection we have looked at the behaviour of the $L=20$ system as a function of Monte Carlo cycles. There are some unexpected non-stastistical behaviour in the plost with the random initial state that we have no explaination for. We have therefor decided to always start with the ordered initial state in the firther calculations. For larger temperatures and $L$ this could mean that we would have to use unneccecary many MC cycles to reach the equilbrium state, but we choose to ignore this in this paper. 

For both temperatures the system stabilized for $~10^5$ MC cycles so we will use a equilibrium time of $t_{eq} = 2\cdot10^5$ MC cycles in this paper. 

In figure \ref{fig:  accepted Trange} we see the number of accepted configurations per spin as a function temperature for the ordered initial state. We start the simulation in the ordered state. It is as axpected that the number of accepted configurations increase with temperarure when the total number of Monte Carlo cycles is much larger ($500000$) than the euilibrium time. This is because when the temperature is low it is very unlikely to flip a spin after the equilibrium state is reached because the state is very close to the ground state. The probability to flip a spin increases as the temperarure increases because the spread in energy is larger, $\sigma_E$ is larger. Thus the number of accepted cycles increase as a function of temperature.

\begin{figure}[htp]
\includegraphics[width=0.8\textwidth]{equilibrium/accepted_Trange_initial1.png}
\caption{Accepted configurations per spin as a function temperature. Top: Ordered initial state (all spinst up). Bottom: Random initial state. The total number of monte carlo cycles is 500000 for the $L=20$ and ordered initial state. }
\centering
\label{fig:  accepted Trange}
\end{figure}

\subsection{Probability}
In this subsection we will look at the probability for a given energy and see it this matches our physical intuition about the system. We will look at the case $L=20$ and the temperatures $T\in \{1.0, 2.0, 1.4\}$. We have used the values for the energy after the system have reached the equilibrium state. This is solved by letting the simulation run for the equilibrium time first and then start to store the energies as the time goes. 

In figure \ref{fig:  probability} we see the probability plotted as a function of energy per spin for the temperatures $T\in \{1.0, 2.0, 1.4\}$ with the coresponding standard deviations $\sigma_E \in \{0.15, 1.71, 2.8 \}$. We see that for the lowest energy it is extremely likely to find the system in the ground state. This coresponds well to the very low standard deviation $\sigma_E = 0.15$. This means that the distance between the mean energy and the other energies are small. Almost every state is in the ground state at low temperatures, then the spread in energy must be small. We see that for the temperature $T=2.0$ the probability is more spread out. It is also more likely to find the system in a slightly higher energy than for the lowest temperature. For the highest temperature $T=2.4$ we see that the probability have shifted even further to higher temperature. The spread of the energies corresponds well with the standard deviation $\sigma_E = 2.8$. It seems reasonable that the system fluctuates around a higher energy when the temperature is higher, and that it closes up to the ground state when the temperature is low. 

\begin{figure}[htp]
\includegraphics[width=0.8\textwidth]{equilibrium/probability.png}
\caption{Probability as a function of energy per spin for the temperatures $T\in \{1.0, 2.0, 1.4\}$ with the coresponding standard deviations $\sigma_E \in \{0.15, 1.71, 2.8 \}$. }
\centering
\label{fig:  probability}
\end{figure}

We have now seen that our system behaves as expected for the ordered initial state. The theoretical values correspond well with the numerical values, the system reaches the equilibrium state and fluctuates around it and the probability as a function of energy behaves as expected for high and low energies. Therefore it seems reasonable to assume that the numerical algorithm is implemented correctly. 

\subsection{Calculations to find $T_c$}
We will now use this algorithm to find the critical temperature for the two dimentional system. We will plot the expextation value for the energy and absolute magnetization, the heat capacity and the suceptibility as a function of temperature for four different values of the lattice, $L\in \{20, 40, 60, 80\}$. We will then try to see if we can see a phase transition and from this calculate the critical temperature for which the phase transition happens. 

In figure \ref{fig:  T_C_20E_M} we see the expextation value for the energy and absolute magnetization, the heat capacity and the suceptibility as a function of temperature for $L=20$. We see that the energy increases as a function of temperature. This is reasonable as seen in the probability plot, for a higher energy the spin state is more random. We also see that the absolute magnetisation is close to 1 for low temperatures. This is natural as then the system is close to the ground state with either spins up or down, $|\mathcal{M}|=1$. For high temeperatures the system is again in a more random state with as many spins up as down. So it is reasonable that the absolute magnetisation goes to zero for higher temperatures. This is also a sign for a phase transition, that a variable 'disappeares'. 

\begin{figure}[htp]
\includegraphics[width=0.8\textwidth]{T_c/first_L20.png}
\caption{The expextation value for the energy and absolute magnetization, the heat capacity and the suceptibility as a function of temperature for $L=20$. The total number of MC cycles is 500000 and the initial state is ordered. }
\centering
\label{fig:  T_C_20E_M}
\end{figure}

\begin{figure}[htp]
\includegraphics[width=0.8\textwidth]{T_c/E_M.png}
\caption{The expextation value for the energy and absolute magnetization as a function of temperature for $L\in \{20, 40, 60, 80\}$. The total number of MC cycles is 500000 and the initial state is ordered. }
\centering
\label{fig:  T_C_E_M}
\end{figure}

In figure \ref{fig:  T_C_E_M} we also see the expextation value for the energy and absolute magnetization as a function of temperature for all lattice sizes, $L\in \{20, 40, 60, 80\}$. We see that the tendency is the same for all the different values of $L$. 

In figure \ref{fig:  T_C_20E_M} we also see that both the heat capacity and the suceptibility have a maximum at about the same temperature. This is also a sign for a phase transition. In figure \ref{fig:  T_C_rest} we see that this maximum becomes sharper for increasing lattice size. 

\begin{figure}[htp]
\includegraphics[width=0.8\textwidth]{T_c/chi_Cv.png}
\caption{The heat capacity and the suceptibility as a function of temperature for $L\in \{20, 40, 60, 80\}$. The total number of MC cycles is 500000 and the initial state is ordered.}
\centering
\label{fig:  T_C_rest}
\end{figure}

The values for the critical temperature read from figure \ref{fig:  T_C_rest} are listed in table \ref{tab: T_c}. The values for $T_c$ varies as a function of $L$ and is read from both the plot of the heat capacity and the suceptibility.

\begin{table}
\centering
\caption{Table of the values for the critical temperature read from figure \ref{fig:  T_C_rest}. The values for $T_c$ varies as a function of $L$ and is read from both the plot of the heat capacity and the suceptibility.}
\begin{tabular}{|c|c|c|}
\hline 
$L$ & $T_C (C_V)$ & $T_c(\chi)$ \\ 
\hline 
20 & 2.32 & 2.34 \\ 
\hline 
40 & 2.29 & 2.31 \\ 
\hline 
60 & 2.28 & 2.30 \\ 
\hline 
80 & 2.27 & 2.3 \\ 
\hline 
\end{tabular}
\label{tab: T_c}
\end{table}

\subsection{Estimating the critical temperature}
We will now use equation \ref{eq:tc} and calculate the best estimate for the critical temperature $T_c$ and compare with Lars Onsagers theoretical result $T_c = 2.269$. This is done in the python script \texttt{calculate\_Tc.py}. We run this python script and obtain the output:
\begin{lstlisting}
Numerical T_c =  2.27
Theoretical T_c =  2.269
Error in estimate =  0.001
\end{lstlisting}
We see that the absolute error in the estimate of the critical temperature is as small as 0.001! 

\section{Discussion and experiences}

This method for obtaining the critical temperature have proved itself to be useful and quite accurate as the error in the estimate was very low! 

But there is room for improvement. We could have run the methods for many times with different seeds in the random generator function for that we would obtain a even larger amount of statistics. It is possible to do this by letting the seed be the negative time in the core in seconds:
\begin{lstlisting}
seed = -time(NULL);
\end{lstlisting}
when we which to produce large amounts of data. 

We could also have let all the negative values of $\Delta E$ go to the metropolis test to aviod the if test for $\Delta E\leq 0$. This is because $e^{-\Delta E/T} > 1$ when $\Delta E\leq 0$ and then the metropolis test $e^{-\Delta E/T} > r$ will never be true. 

We could also have regulated the equilibrium time so that it corresponded to the lattice seze $L$ and the temperature. As we could also have chosen the initial state according to the temperature. Both cases would have saved a lot of execution time. But that was not a priority in this project because the lattice sizes run are not so huge. Imagine running for lattice sizes of order a 1000...

Either way we sould be happy about estimating the critical temperature with quite good precition.

\end{document}






